\documentclass[UTF8]{ctexart}
\usepackage{ctex}
\usepackage{amsmath}
\usepackage{amsthm}
\usepackage{geometry}
\geometry{left=2.5cm,right=2.5cm,top=2.5cm,bottom=2.5cm}
\usepackage{amssymb}
\usepackage{indentfirst}
\usepackage{graphicx}
\usepackage{subfigure}
\usepackage{listings}
\usepackage{xcolor}
\usepackage{float}
\usepackage{algorithm}  
\usepackage{algorithmicx}  
\usepackage{longtable}
\usepackage{fancyhdr}
\usepackage{appendix}
\usepackage{enumitem}
\usepackage{abstract}
\usepackage{multirow}
\pagestyle{fancy}
\lfoot{}%这条语句可以让页码出现在下方
\theoremstyle{plain}
\newtheorem{thm}{Theorem}[section]
\newtheorem{lem}[thm]{Lemma}
\newtheorem{prop}[thm]{Proposition}
\newtheorem{cor}[thm]{Corollary}

\theoremstyle{definition}
\newtheorem{defn}{Definition}[section]

\theoremstyle{remark}
\newtheorem*{rem}{Remark}
\newtheorem{eg}{Example}[section]
\newcommand{\supp}{\text{supp}}
\newcommand{\Rn}{\mathbb{R}^{n}}
\newcommand{\dif}{\mathrm{d}}
\newcommand{\avg}[1]{\left\langle #1 \right\rangle}
\newcommand{\difFrac}[2]{\frac{\dif #1}{\dif #2}}
\newcommand{\pdfFrac}[2]{\frac{\partial #1}{\partial #2}}
\newcommand{\OFL}{\mathrm{OFL}}
\newcommand{\UFL}{\mathrm{UFL}}
\newcommand{\fl}{\mathrm{fl}}
\newcommand{\op}{\odot}
\newcommand{\cp}{\cdot}
\newcommand{\Eabs}{E_{\mathrm{abs}}}
\newcommand{\Erel}{E_{\mathrm{rel}}}
\newcommand{\DR}{\mathcal{D}_{\widetilde{LN}}^{n}}
\newcommand{\add}[2]{\sum_{#1=1}^{#2}}
\newcommand{\innerprod}[2]{\left<#1,#2\right>}
\newcommand\tbbint{{-\mkern -16mu\int}}
\newcommand\tbint{{\mathchar '26\mkern -14mu\int}}
\newcommand\dbbint{{-\mkern -19mu\int}}
\newcommand\dbint{{\mathchar '26\mkern -18mu\int}}
\newcommand\bint{
{\mathchoice{\dbint}{\tbint}{\tbint}{\tbint}}
}
\newcommand\bbint{
{\mathchoice{\dbbint}{\tbbint}{\tbbint}{\tbbint}}
}
\title{Finite Volume Design}
\author{Shuang Hu}
\begin{document}
\maketitle
\section{问题描述}
设计四阶精度的有限体积算法求解\textbf{对流扩散方程}的初边值问题和\textbf{不可压Navier-Stokes方程}的周期边界问题。方程的表示形式具体如下:

对流扩散方程:
\begin{equation}
    \left\{
        \begin{aligned}
            \pdfFrac{\phi}{t}&=-\nabla\cdot(\mathbf{u}\phi)+\nu\Delta\phi+f,\\
            \phi(\mathbf{x},0)&=g_{1}(\mathbf{x}),\mathbf{x}\in\Omega,\\
            \phi(\mathbf{x},t)&=g_{2}(\mathbf{x},t),\mathbf{x}\in\partial\Omega.\\
        \end{aligned}
    \right.
\end{equation}
如果是Neumann边界条件,第三个表达式则改为
\begin{equation}
    \pdfFrac{\phi}{\mathbf{n}}=g_{2}(\mathbf{x},t).
\end{equation}

周期边界的INSE,区域$\Omega:=[0,1]^2$:
\begin{equation}
    \left\{
        \begin{aligned}
            \pdfFrac{\mathbf{u}}{t}+\mathbf{u}\cdot\nabla\mathbf{u}&=\mathbf{g}-\nabla p+\nu\Delta\mathbf{u},\\
            \nabla\cdot\mathbf{u}&=0,\mathbf{x}\in\Omega,\\
            \mathbf{u}(x,y,t)&=\mathbf{u}(x+1,y,t).\\
            \mathbf{u}(x,y,0)&=\mathbf{u}_{0}(x,y)\\
        \end{aligned}
    \right.
\end{equation}
在本次作业中,需要实现对流扩散方程在\textbf{Dirichlet}和\textbf{Neumann}两种边界条件下的求解,其中时间积分方法利用\textbf{ERK-ESDIRK IMEX Runge-Kutta格式},并且在近似\textbf{Leray-Helmholtz投影算子}时,需要采用多重网格算法。
\section{底层程序}
底层的数据结构和数值算法沿用先前组里求解Navier-Stokes方程的软件包,在本程序中需要用到的是以下内容:
\begin{itemize}
    \item \texttt{class Vec}:用来表示空间中的点。
    \item \texttt{class Tensor}:用来存储体平均值,表示系数矩阵等。
    \item \texttt{class RowSparse}:用于存储稀疏矩阵。
    \item \texttt{class Box}:用于网格离散。
    \item \texttt{class RectDomain}:用于表示问题区域(矩形)。
    \item \texttt{numlib.h}:一些常用的数值算法,这里会多次用到数值积分程序段。
    \item 约定一些符号表示:
    \begin{itemize}
        \item \texttt{template<int Dim>} 表示问题区域的维数。
        \item \texttt{using rVec=Vec<Real,Dim>}
    \end{itemize}
\end{itemize}
\section{FuncFiller}
\subsection{class VectorFunction}
\begin{itemize}
    \item 函数$\mathbb{R}^{\text{Dim}_{1}}\rightarrow\mathbb{R}^{\text{Dim}_{2}}$的基类,用于表示方程的初值/边值信息,或者是右端项。
    \item \textbf{模板:} \texttt{template<int Dim1,int Dim2>}:
    
    \texttt{Dim1}和\texttt{Dim2}分别表示定义域和值域所在的空间维数。其中,如果\texttt{Dim2=1},则代表该函数为标量函数。
    \item \textbf{成员函数:}
    \begin{enumerate}
    \item \texttt{virtual const Vec<Dim2> operator()(const Vec<Dim1>\& pt) const = 0;}

      \textbf{public成员函数}

      \textbf{输入:} \texttt{pt}表示\texttt{Dim1}维空间中的一个点。

      \textbf{输出:} 该函数在\texttt{pt}点处的取值。

      \textbf{作用:} 计算函数在一个离散点上的值。纯虚函数,需要在继承类中具体实现。
    \end{enumerate}
\end{itemize}

\subsection{class FuncFiller}
\begin{itemize}
    \item 将所给函数的积分平均值填充到离散网格中。
    \item \textbf{模板:}\texttt{template<int Dim>:}
    
    \texttt{Dim}表示问题空间的维数。在本次作业中,该模板参数取2,下同。

    \item \texttt{using rVec=Vec<Real,Dim>}。
    \item \textbf{成员变量:}
    \begin{enumerate}
        \item \texttt{RectDomain<Dim> domain:} \textbf{private成员变量},表示需要填充的均匀网格。
    \end{enumerate}

    \item \textbf{成员函数:}
    \begin{enumerate}
        \item \texttt{FuncFiller(const RectDomain<Dim>\& adomain);}
        
        \textbf{public成员函数}
        
        \textbf{输入:} 需要进行函数值填充的问题区域,要求为矩形区域。

        \textbf{作用:} 构造函数,记录问题区域的信息。
        \item \texttt{template<int Dim2>}
        
        \texttt{Vec<Real,Dim2> Quadrature(const rVec\& lo,const rVec\& hi,\\const VectorFunction<Dim,Dim2>* func) const;}
        
        \textbf{private成员函数}
        
        \textbf{输入:} \texttt{lo}表示正方体区域左下角,\texttt{hi}表示区域右上角,\texttt{func}表示需要求积分的函数。

        \textbf{输出:}函数\texttt{func}在网格\texttt{[lo,hi]}上积分平均的近似值。

        \textbf{作用:}计算函数在一个网格上积分平均的近似值,即体平均值。


        \item 
        \texttt{template<int Dim2>}

        \texttt{void fill(Tensor<Vec<Real,Dim2>,Dim>\& target, const VectorFunction<Dim,Dim2>* func) const;} 
        
        \textbf{public 成员函数}

        \textbf{输入:} \texttt{target}为待填充的\texttt{Dim}维\texttt{Tensor}; \texttt{func}指向一个函数。

        \textbf{作用:} 将函数\texttt{*func}在每个离散网格上的积分平均值填入\texttt{target}。

    \end{enumerate}
\end{itemize}
\subsection{class GhostFiller}
\texttt{enum BdryType\{ Dirichlet=0, Neumann=1, Periodic=2\};}
\begin{itemize}
    \item 利用已知的边界条件填充\texttt{Ghost Cell}。
    \item \textbf{模板:} \texttt{template<int Dim,BdryType BCType,class T>:}
    
    \texttt{Dim}表示空间维数,\texttt{BCType}表示边界条件种类,\texttt{T}表示输入张量的数据类型。

    \texttt{BCType}表示边值条件的种类是\texttt{Dirichlet}或\texttt{Neumann}。
    \item \textbf{成员变量:}
    \begin{enumerate}
        \item \texttt{RectDomain<Dim> domain:} \textbf{private成员变量},表示需要填充\texttt{Ghost Cell}的区域。
        \item \texttt{Tensor<T,Dim> bdryCond:} \textbf{private成员变量},表示边界处的面平均信息。
        \item \texttt{int nGhost:} \textbf{private成员变量},表示需要填充的\texttt{Ghost Cell}层数。
    \end{enumerate}
    \item \textbf{成员函数:}
    \begin{enumerate}
        \item \texttt{template<int Dim2>}
        
        \texttt{Vec<Real,Dim2> FaceAverage(const rVec\& lo, const rVec\& hi, Real t0, \\const VectorFunction<Dim+1,Dim2>* func) const;}
        
        \textbf{private成员函数}

        \textbf{输入:} \texttt{lo}表示控制面的左下角,\texttt{hi}表示控制面的右上角,\texttt{func}指向需要求积分的函数,\texttt{t0}表示边界函数的时间参量。

        \textbf{输出:} 函数\texttt{func}在控制面\texttt{[lo,hi]}上的积分平均值。

        \textbf{作用:} 计算标量边界函数$f(\mathbf{x},t_{0})$在控制面上的积分平均。

        \item \texttt{GhostFiller(const RectDomain<Dim>\& aDomain, int numGhost);}
        
        \textbf{public成员函数}

        \textbf{输入:} \texttt{aDomain}表示需要进行\texttt{Ghost Cell}填充的区域,\texttt{atype}表示边界条件的种类, 
        
        \texttt{numGhost}表示需要填写\texttt{Ghost Cell}的层数。

        \textbf{作用:}构造函数,记录区域信息,边界条件信息,并初始化\texttt{bdryCond}。

        \item 
        \texttt{template<int Dim2>}

        \texttt{void fillBdry(const VectorFunction<Dim+1,Dim2>* func, Real t0);}
        
        \textbf{public成员函数}

        \textbf{输入:} \texttt{func}表示边值条件,\texttt{t0}表示初始时刻。

        \textbf{作用:}利用边值条件计算$t_{0}$时刻区域边界控制面的平均值,并填充至\texttt{bdryCond}。

        \item \texttt{void fillGhost(Tensor<T,Dim>\& res) const;}
        
        \textbf{输入:} \texttt{res}表示需要填充\texttt{Ghost Cell}张量。

        \textbf{作用:}利用\texttt{res}中原有的体平均值和计算出的面平均值,向外填充\texttt{Ghost Cell},更新\texttt{res}的值。
    \end{enumerate}
\end{itemize}
\section{MultiGrid}
这一部分的功能是,在\textbf{填写完Ghost Cell}的情况下,利用多重网格算法求解线性方程组$Ax=b$。这里系数矩阵$A$是由一个椭圆算子导出的。

具体地,几乎可以一字不改地应用之前"微分方程数值解"课程中分享的那个多重网格设计,此处就不再重复了。
\section{SpacialOperator}
\subsection{class SpacialOp}
\begin{itemize}
    \item 虚基类,用于表示离散的空间算子。
    \item \textbf{模板:}\texttt{template<int Dim,class T1,class T2>:}
    
    \texttt{Dim}表示空间维数,\texttt{T1}表示算子作用前的数据类型,\texttt{T2}表示算子作用后的数据类型。

    \item \textbf{成员变量:}
    \begin{enumerate}
        \item \texttt{RectDomain<Dim> domain:} \textbf{protected成员变量},表示需要填充\texttt{Ghost Cell}的区域。
        \item \texttt{int nGhost:} \textbf{protected成员变量},表示需要填充\texttt{Ghost Cell}的层数。
    \end{enumerate}

    \item \textbf{成员函数:}
    \begin{enumerate}
        \item \texttt{SpacialOp(const RectDomain<int>\& aDomain,int numGhost);}
        
        \textbf{public成员函数}

        \textbf{输入:} \texttt{aDomain}表示问题区域,\texttt{numGhost}表示需要填充\texttt{Ghost Cell}的层数。

        \textbf{作用:} 构造函数,记录空间算子所需的信息。

        \item \texttt{void getFaceAvg(const Tensor<T1,Dim>\& origin, Tensor<T1,Dim>\& res,\\ int dim);}
        
        \textbf{public成员函数}

        \textbf{输入:}\texttt{origin}表示\textbf{填写过Ghost Cell的}控制体平均值张量,\texttt{res}表示控制面平均张量的近似值,\texttt{dim}表示在第\texttt{dim}维度上求解面平均。

        \textbf{作用:}利用体平均来近似计算面平均。

        \item \texttt{virtual void apply(const Tensor<T1,Dim>\& origin, Tensor<T2,Dim>\& res) = 0;}
        
        \textbf{public成员函数}

        \textbf{输入:}\texttt{origin}表示控制体平均张量, \texttt{res}表示在\texttt{origin}上作用离散算子之后得到的控制体平均值张量。

        \textbf{作用:}用于描述空间离散算子的作用结果。
    \end{enumerate}
\end{itemize}
\subsection{class Laplacian}
\begin{itemize}
    \item 用于表示拉普拉斯算子的四阶有限体积离散。
    \item \textbf{模板:} \texttt{template<int Dim>}
    \item \textbf{继承:} \texttt{class Laplacian:public SpacialOp<Dim,Real,Real>}
    \item \textbf{成员函数:}
    \begin{enumerate}
        \item \texttt{void apply(const Tensor<Real,Dim>\& origin, Tensor<Real,Dim>\& res);}
        
        \textbf{public成员函数}

        \textbf{输入:}\texttt{origin}表示控制体上的体平均张量,\texttt{res}表示作用拉普拉斯算子后控制体上的体平均张量。

        \textbf{作用:}对拉普拉斯算子进行离散化。
    \end{enumerate}
\end{itemize}
\subsection{class Divergent}
\begin{itemize}
    \item 用于表示散度算子的四阶有限体积离散
    \item \textbf{模板:} \texttt{template<int Dim>}
    \item \texttt{using rVec=Vec<Real,Dim>;}
    \item \textbf{继承:} \texttt{class Divergent:public SpacialOp<Dim,rVec,Real>}
    \item \textbf{成员函数:} 
    \begin{enumerate}
        \item \texttt{void apply(const Tensor<rVec,Dim>\& origin, Tensor<Real,Dim>\& res);}
        
        \textbf{public成员函数}
        
        \textbf{输入:}\texttt{origin}表示一个向量场的体平均值,\texttt{res}表示该向量场上作用离散散度算子后的体平均值。

        \textbf{作用:}对散度算子进行离散化。
    \end{enumerate}
\end{itemize}
\subsection{class Gradient}
\begin{itemize}
    \item 用于表示梯度算子的四阶有限体积离散
    \item \textbf{模板:}\texttt{template<int Dim>}
    \item \texttt{using rVec=Vec<Real,Dim>}
    \item \textbf{继承:}\texttt{class Gradient:public SpacialOp<Dim,Real,rVec>}
    \item \textbf{成员函数:}
    \begin{enumerate}
        \item \texttt{void apply(const Tensor<Real,Dim>\& origin, Tensor<rVec,Dim>\& res);}
        
        \textbf{public成员函数}

        \textbf{输入:}\texttt{origin}表示一个标量场的体平均值,\texttt{res}表示在该标量场上作用离散梯度算子后得到的体平均值。

        \textbf{作用:}对梯度算子进行离散化。
    \end{enumerate}
\end{itemize}
\subsection{class InnerProduct}
\begin{itemize}
    \item 用于求解内积算子面平均值的离散化。
    \item \textbf{模板:} \texttt{template<int Dim>}
    \item \textbf{继承:} \texttt{class InnerProduct:public SpacialOp<Dim,Real,Real>}
    \item \textbf{成员变量:}
    \begin{enumerate}
        \item \texttt{int dim:} \textbf{private成员变量,}用于表示当前我们所关心的控制面。
        \item \texttt{Tensor<Real,Dim> phi:} \textbf{private成员变量,}表示内积表达式中的其中一个函数。
    \end{enumerate}
    \item \textbf{成员函数:}
    \begin{enumerate}
        \item \texttt{InnerProduct(const RectDomain<Dim>\& aDomain, int numGhost,const Tensor<Real,Dim>\& aphi, int adim);}
        
        \textbf{public成员函数}

        \textbf{输入:}\texttt{aDomain,numGhost}用于初始化基类,\texttt{adim,aphi}用于初始化本派生类。

        \textbf{作用:}构造函数。

        \item \texttt{void apply(const Tensor<Real,Dim>\& origin, Tensor<Real,Dim>\& res);}
        
        \textbf{public成员函数}

        \textbf{输入:}\texttt{origin}表示一个标量场在\texttt{dim}维上的\textbf{面平均值},\texttt{res}表示在该控制面上\texttt{origin}和\texttt{phi}作离散内积后的面平均值。
    \end{enumerate}
\end{itemize}
\subsection{class AD\_Advection}
\begin{itemize}
    \item 在对流扩散方程中,用于实现对流项的离散化。
    \item \textbf{模板:}\texttt{template<int Dim>}
    \item \textbf{继承:}\texttt{class Advection:public SpacialOp<Dim,Real,Real>}
    \item \texttt{using rVec=Vec<Real,Dim>}
    \item \textbf{成员变量:}
    \begin{enumerate}
        \item \texttt{Tensor<rVec,Dim> u:}\textbf{private成员变量},表示已知的速度场。
    \end{enumerate}
    \item \textbf{成员函数:}
    \begin{enumerate}
        \item \texttt{AD\_Advection(const RectDomain<Dim>\& aDomain, int numGhost,\\ const Tensor<rVec,Dim>\& vel);}
        
        \textbf{public成员函数}

        \textbf{输入:} \texttt{aDomain,numGhost}用于构造基类,\texttt{vel}表示已知的速度场,用于构造派生类。

        \textbf{作用:}构造函数。

        \item \texttt{void apply(const Tensor<Real,Dim>\& origin, Tensor<Real,Dim>\& res);}
        
        \textbf{public成员函数}

        \textbf{输入:} \texttt{origin}表示$\phi$在控制体上的体平均值,\texttt{res}表示$\nabla\cdot(\textbf{u}\phi)$在控制体上体平均值的近似。

        \textbf{作用:} 求解对流扩散方程中的对流项。
    \end{enumerate}
\end{itemize}
\subsection{class INSE\_Advection}
\begin{itemize}
    \item 在不可压Navier-Stokes方程中,实现对流项的离散化。
    \item \textbf{模板:}\texttt{template<int Dim>}
    \item \textbf{继承:}\texttt{class INSE\_Advection:public SpacialOp<Dim,Vec<Real,Dim>,Vec<Real,Dim>>}
    \item \texttt{using rVec=Vec<Real,Dim>}.
    \item \textbf{成员函数:}
    
    \begin{enumerate}
        \item \texttt{void apply(const Tensor<rVec,Dim>\& origin, Tensor<rVec,Dim>\& res);}
        
        \textbf{public成员函数}

        \textbf{输入:}\texttt{origin}表示输入的速度场的体平均值,\texttt{res}表示计算得到的$\nabla\cdot(\mathbf{u}\mathbf{u})$的近似。

        \textbf{作用:}近似计算对流项。
    \end{enumerate}
\end{itemize}
\subsection{class Inv\_Laplacian}
\begin{itemize}
    \item 用于近似拉普拉斯算子的逆。
    \item TBD
\end{itemize}
\subsection{class Projection}
\begin{itemize}
    \item 用于实现投影算子的近似。表达式:$P=I-DL^{-1}G$。
    \item \textbf{模板:}\texttt{template<int Dim>}
    \item \textbf{继承:}\texttt{class Projection:public SpatialOp<Dim,Vec<Real,Dim>,Vec<Real,Dim>>;}
    \item \texttt{using rVec=Vec<Real,Dim>;}
    \item \textbf{成员函数:}
    \begin{enumerate}
        \item \texttt{void apply(const Tensor<rVec,Dim>\& origin, Tensor<rVec,Dim>\& res);}
        
        \textbf{public成员函数}

        \textbf{输入:} \texttt{origin}为给定的速度场,\texttt{res}为作用投影算子之后的速度场。

        \textbf{作用:} 进行Leray-Helmholtz投影算子的数值模拟。
    \end{enumerate}
\end{itemize}
\section{TimeIntegration}
\subsection{class RungeKutta}
\begin{itemize}
    \item 进行一步Runge-Kutta迭代,求解初值问题
    \begin{equation}
        \left\{
            \begin{aligned}
        \difFrac{\mathbf{x}}{t}&=f(\mathbf{x},t),\\
        \mathbf{x}(t_{0})&=\mathbf{x}_{0}.
            \end{aligned}
        \right.
    \end{equation}
    \item \textbf{模板:} \texttt{template<int Dim,class DType>:} \texttt{DType}表示初值问题变量的数据类型。
    \item \textbf{成员变量:}
    \begin{enumerate}
        \item \texttt{Tensor<Real,Dim> butlar:} \textbf{private成员变量},记录Runge-Kutta法对应的Butlar表。
        \item \texttt{RectDomain<Dim> domain:} \textbf{private成员变量},记录区域信息。
    \end{enumerate}
    \item \textbf{成员函数:}
    \begin{enumerate}
        \item \texttt{virtual void getRHS(const Tensor<DType,Dim>\& input, Tensor<DType,Dim>\& rhs, \\Real t) = 0;}
        
        \textbf{public成员函数}

        \textbf{输入:}\texttt{input}表示目前的体平均张量值,\texttt{rhs}表示在$t$时刻的右端项。

        \textbf{作用:}用于描述$f(\mathbf{x},t)$,纯虚函数,依赖于派生类的具体实现,与算子离散方式有关。
        \item \texttt{void apply(Tensor<DType,Dim>\& res, const Tensor<DType,Dim>\& input, Real t0,\\ Real k);}
        
        \textbf{public成员函数}

        \textbf{输入:} \texttt{input}表示初始值$\textbf{x}(t_{0})$,\texttt{rhs}表示近似值$\textbf{x}(t_{0}+k)$, \texttt{t0}表示初始时刻, \texttt{k}表示时间间隔。

        \textbf{作用:}在已知$\mathbf{x}(t_{0})=\mathbf{x}_{0}$的情况下,利用\texttt{Runge-Kutta}算法求解$\mathbf{x}(t_{0}+k)$。
    \end{enumerate}
\end{itemize}
\subsection{class ADSolver}
\begin{itemize}
    \item 记录对流扩散方程的信息,并进行求解操作。
    \item \textbf{模板:}\texttt{template<int Dim,BdryType BCType>}
    \item \textbf{继承:}\texttt{class ADSolver:public RungeKutta<Dim,Real>}
    \item \textbf{成员变量:}
    \begin{enumerate}
        \item \texttt{VectorFunction<Dim+1,1>* f:} \textbf{private成员变量},记录外力项$f$。
        \item \texttt{VectorFunction<Dim+1,Dim>* u:} \textbf{private成员变量},表示已知的速度场$\mathbf{u}$。
        \item \texttt{VectorFunction<Dim,1>* init\_value:} \textbf{private成员变量},表示方程的初值信息$\phi(\mathbf{x},t_{0})$。
        \item \texttt{VectorFunction<Dim+1,1>* bdry\_value:} \textbf{private成员变量},表示方程的边值信息。
        \item \texttt{Real nu:} \textbf{private成员变量},表示扩散项前面的系数。
        \item \texttt{RectDomain<Dim> domain:} \textbf{private成员变量},表示问题区域。
        \item \texttt{FuncFiller<Dim> init\_filler:} \textbf{private成员变量},用于填写初值的体平均。
        \item \texttt{GhostFiller<Dim,BCType> ghost\_filler:} \textbf{private成员变量},用于填写\texttt{Ghost Cell}。
        \item \texttt{Tensor<Real,Dim> res:} \textbf{private成员变量},用于表示$\phi$的体平均求解结果。
        \item \texttt{Real t0:} \textbf{private成员变量},用于表示初始时刻。
        \item \texttt{Real te:} \textbf{private成员变量},用于表示迭代终止时刻。
        \item \texttt{Real dt:} \textbf{private成员变量},用于表示时间间隔。
        \item \texttt{Laplacian<Dim> lap:} \textbf{private成员变量},用于表示拉普拉斯算子离散。
        \item \texttt{Advection<Dim> adv:} \textbf{private成员变量},用于表示对流算子的离散。
    \end{enumerate}
    \item \textbf{成员函数:}
    \begin{enumerate}
        \item \texttt{ADSolver(const RectDomain<Dim>\& aDomain);} 
        
        \textbf{public成员函数}

        \textbf{输入:}\texttt{aDomain}表示问题区域。

        \textbf{作用:}构造函数,明确求解的问题区域,并依此对各离散算子进行初始化。

        \item \texttt{void set\_param(VectorFunction<Dim+1,1>* force,VectorFunction<Dim+1,Dim>* vel,\\VectorFunction<Dim,1>* init,VectorFunction<Dim+1,1>* bdry,Real anu,\\Real init\_time, Real final\_time, Real dt);}
        
        \textbf{public成员函数}

        \textbf{作用:}初始化方程求解的各个变量。

        \item \texttt{void apply(Tensor<Real,Dim>\& input, Tensor<Real,Dim>\& rhs, Real t);}

        \textbf{public成员函数}

        \textbf{作用:}实现单步\texttt{Runge-Kutta}法迭代。

        \item \texttt{void solve();}

        \textbf{public成员函数}

        \textbf{作用:}最终实现方程组求解。

        \item \texttt{void draw();}
        
        \textbf{public成员函数}

        \textbf{作用:}基于最终解出的体平均张量\texttt{res}信息,对求解结果进行可视化。

    \end{enumerate}
\end{itemize}
\subsection{class INSESolver}
\begin{itemize}
    \item 记录周期边界条件下不可压Navier-Stokes方程的相关信息,并进行有限体积法求解。
    \item \textbf{模板:}\texttt{template<int Dim>}
    \item \textbf{继承:}\texttt{class INSESolver:public RungeKutta<Dim,Vec<Real,Dim>>}
    \item \texttt{using rVec=Vec<Real,Dim>}
    \item \textbf{成员变量:}
    \begin{enumerate}
        \item \texttt{VectorFunction<Dim+1,Dim>* g}: \textbf{private成员变量},表示外力项。
        \item \texttt{VectorFunction<Dim,Dim>* u0}: \textbf{private成员变量},表示速度场的初始信息。
        \item \texttt{Real Re}: \textbf{private成员变量},表示雷诺数。
        \item \texttt{RectDomain<Dim> domain}: \textbf{private成员变量},表示问题区域。
        \item \texttt{Projection<Dim> proj}: \textbf{private成员变量},表示投影算子。
        \item \texttt{Laplacian<Dim> lap}: \textbf{private成员变量},表示拉普拉斯算子。
        \item \texttt{INSE\_Advection<Dim> adv}: \textbf{private成员变量},表示对流算子。
        \item \texttt{FuncFiller<Dim> init\_filler:} \textbf{private成员变量},用于填写初值体平均。
        \item \texttt{GhostFiller<Dim,Periodic> ghost\_filler:} \textbf{private成员变量},用于填写\texttt{Ghost Cell}。
        \item \texttt{Tensor<rVec,Dim> res:} \textbf{private成员变量},用于表示速度场体平均的求解结果。
        \item \texttt{Real t0}: \textbf{private成员变量},用于表示初始时刻。
        \item \texttt{Real te}: \textbf{private成员变量},用于表示迭代终止时刻。
        \item \texttt{Real dt}: \textbf{private成员变量},用于表示时间间隔。
    \end{enumerate}
    \item \textbf{成员函数:}
    \begin{enumerate}
        \item \texttt{INSESolver(const RectDomain<Dim>\& aDomain);}
        
        \textbf{public成员函数}

        \textbf{输入:} \texttt{aDomain}表示问题区域。

        \textbf{作用:} 构造函数,明确求解的问题区域,并依此对各离散算子进行初始化。

        \item \texttt{void set\_param(VectorFunction<Dim+1,Dim>* force,VectorFunction<Dim,Dim>* init\_vel,Real Reynold);}
        
        \textbf{public成员函数}

        \textbf{作用:}初始化方程求解的各个变量。

        \item \texttt{void apply(Tensor<rVec,Dim>\& input, Tensor<rVec,Dim>\& rhs, Real t);}
        
        \textbf{public成员函数}

        \textbf{作用:}实现单步Runge-Kutta迭代。

        \item \texttt{void solve();}
        
        \textbf{public成员函数}

        \textbf{作用:}最终实现方程求解。
    \end{enumerate}
\end{itemize}
\end{document}