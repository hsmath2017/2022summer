\documentclass[UTF8]{ctexart}
\usepackage{ctex}
\usepackage{amsmath}
\usepackage{amsthm}
\usepackage{geometry}
\geometry{left=2.5cm,right=2.5cm,top=2.5cm,bottom=2.5cm}
\usepackage{amssymb}
\usepackage{indentfirst}
\usepackage{graphicx}
\usepackage{subfigure}
\usepackage{listings}
\usepackage{xcolor}
\usepackage{float}
\usepackage{algorithm}  
\usepackage{algorithmicx}  
\usepackage{longtable}
\usepackage{fancyhdr}
\usepackage{appendix}
\usepackage{enumitem}
\usepackage{abstract}
\usepackage{multirow}
\pagestyle{fancy}
\lfoot{}%这条语句可以让页码出现在下方
\theoremstyle{plain}
\newtheorem{thm}{Theorem}[section]
\newtheorem{lem}[thm]{Lemma}
\newtheorem{prop}[thm]{Proposition}
\newtheorem{cor}[thm]{Corollary}

\theoremstyle{definition}
\newtheorem{defn}{Definition}[section]

\theoremstyle{remark}
\newtheorem*{rem}{Remark}
\newtheorem{eg}{Example}[section]
\newcommand{\supp}{\text{supp}}
\newcommand{\Rn}{\mathbb{R}^{n}}
\newcommand{\dif}{\mathrm{d}}
\newcommand{\avg}[1]{\left\langle #1 \right\rangle}
\newcommand{\difFrac}[2]{\frac{\dif #1}{\dif #2}}
\newcommand{\pdfFrac}[2]{\frac{\partial #1}{\partial #2}}
\newcommand{\OFL}{\mathrm{OFL}}
\newcommand{\UFL}{\mathrm{UFL}}
\newcommand{\fl}{\mathrm{fl}}
\newcommand{\op}{\odot}
\newcommand{\cp}{\cdot}
\newcommand{\Eabs}{E_{\mathrm{abs}}}
\newcommand{\Erel}{E_{\mathrm{rel}}}
\newcommand{\DR}{\mathcal{D}_{\widetilde{LN}}^{n}}
\newcommand{\add}[2]{\sum_{#1=1}^{#2}}
\newcommand{\innerprod}[2]{\left<#1,#2\right>}
\newcommand\tbbint{{-\mkern -16mu\int}}
\newcommand\tbint{{\mathchar '26\mkern -14mu\int}}
\newcommand\dbbint{{-\mkern -19mu\int}}
\newcommand\dbint{{\mathchar '26\mkern -18mu\int}}
\newcommand\bint{
{\mathchoice{\dbint}{\tbint}{\tbint}{\tbint}}
}
\newcommand\bbint{
{\mathchoice{\dbbint}{\tbbint}{\tbbint}{\tbbint}}
}
\title{Finite Volume Design}
\author{Shuang Hu}
\begin{document}
\maketitle
\section{问题描述}
设计四阶精度的有限体积算法求解\textbf{对流扩散方程}的初边值问题和\textbf{不可压Navier-Stokes方程}的周期边界问题。方程的表示形式具体如下:

对流扩散方程:
\begin{equation}
    \left\{
        \begin{aligned}
            \pdfFrac{\phi}{t}&=-\nabla\cdot(\mathbf{u}\phi)+\nu\Delta\phi+f,\\
            \phi(\mathbf{x},0)&=g_{1}(\mathbf{x}),\mathbf{x}\in\Omega,\\
            \phi(\mathbf{x},t)&=g_{2}(\mathbf{x},t),\mathbf{x}\in\partial\Omega.\\
        \end{aligned}
    \right.
\end{equation}
如果是Neumann边界条件,第三个表达式则改为
\begin{equation}
    \pdfFrac{\phi}{\mathbf{n}}=g_{2}(\mathbf{x},t).
\end{equation}

周期边界的INSE,区域$\Omega:=[0,1]^2$:
\begin{equation}
    \left\{
        \begin{aligned}
            \pdfFrac{\mathbf{u}}{t}+\mathbf{u}\cdot\nabla\mathbf{u}&=\mathbf{g}-\nabla p+\nu\Delta\mathbf{u},\\
            \nabla\cdot\mathbf{u}&=0,\mathbf{x}\in\Omega,\\
            \mathbf{u}(x,y,t)&=\mathbf{u}(x+1,y,t).\\
            \mathbf{u}(x,y,0)&=\mathbf{u}_{0}(x,y)\\
        \end{aligned}
    \right.
\end{equation}
在本次作业中,需要实现对流扩散方程在\textbf{Dirichlet}和\textbf{Neumann}两种边界条件下的求解,其中时间积分方法利用\textbf{ERK-ESDIRK IMEX Runge-Kutta格式},并且在近似\textbf{Leray-Helmholtz投影算子}时,需要采用多重网格算法。
\section{底层程序}
底层的数据结构和数值算法沿用先前组里求解Navier-Stokes方程的软件包,在本程序中需要用到的是以下内容:
\begin{itemize}
    \item \texttt{class Vec}:用来表示空间中的点。
    \item \texttt{class Tensor}:用来存储体平均值,表示系数矩阵等。
    \item \texttt{class RowSparse}:用于存储稀疏矩阵。
    \item \texttt{class Box}:用于网格离散。
    \item \texttt{class RectDomain}:用于表示问题区域(矩形)。
    \item \texttt{numlib.h}:一些常用的数值算法,这里会多次用到数值积分程序段。
    \item 约定一些符号表示:
    \begin{itemize}
        \item \texttt{template<int Dim>} 表示问题区域的维数。
        \item \texttt{using rVec=Vec<Real,Dim>}
    \end{itemize}
\end{itemize}
\section{class VectorFunction}
\begin{itemize}
    \item 函数$\mathbb{R}^{\text{Dim}_{1}}\rightarrow\mathbb{R}^{\text{Dim}_{2}}$的基类,用于表示方程的初值/边值信息,或者是右端项。
    \item \textbf{模板:} \texttt{template<int Dim1,int Dim2>}:
    
    \texttt{Dim1}和\texttt{Dim2}分别表示定义域和值域所在的空间维数。
    \item \textbf{成员函数:}
    \begin{enumerate}
    \item \texttt{virtual const Vec<Dim2> operator()(const Vec<Dim1>\& pt) const = 0;}

      \textbf{public成员函数}

      \textbf{输入:} \texttt{pt}表示\texttt{Dim1}维空间中的一个点。

      \textbf{输出:} 该函数在\texttt{pt}点处的取值。

      \textbf{作用:} 计算函数在一个离散点上的值。纯虚函数,需要在继承类中具体实现。
    \end{enumerate}
\end{itemize}
\section{class FuncFiller}
\begin{itemize}
    \item 将所给函数的积分平均值填充到离散网格中。
    \item \textbf{模板:}\texttt{template<int Dim>:}
    
    \texttt{Dim}表示问题空间的维数。在本次作业中,该模板参数取2,下同。

    \item \textbf{成员变量:}
    \begin{enumerate}
        \item \texttt{RectDomain<Dim> domain:} \textbf{private成员变量},表示需要填充的均匀网格。
    \end{enumerate}

    \item \textbf{成员函数:}
    \begin{enumerate}
        \item \texttt{FuncFiller(const RectDomain<Dim>\& adomain);}
        
        \textbf{public成员函数}
        
        \textbf{输入:} 需要进行函数值填充的问题区域,要求为矩形区域。

        \textbf{作用:} 构造函数,记录问题区域的信息。
        \item \texttt{Real Quadrature(const rVec\& lo,const rVec\& hi,\\const VectorFunction<Dim>* func) const;}
        
        \textbf{private成员函数}
        
        \textbf{输入:} \texttt{lo}表示正方体区域左下角,\texttt{hi}表示区域右上角,\texttt{func}表示需要求积分的函数。

        \textbf{输出:}函数\texttt{func}在网格\texttt{[lo,hi]}上积分平均的近似值。

        \textbf{作用:}计算函数在一个网格上积分平均的近似值,即体平均值。

        \item \texttt{void fill(Tensor<Real,Dim>\& target, const VectorFunction<Dim>* func) const;} 
        
        \textbf{public 成员函数}

        \textbf{输入:} \texttt{target}为待填充的\texttt{Dim}维\texttt{Tensor}; \texttt{func}为用来填充函数体平均值的指针。

        \textbf{作用:} 将函数\texttt{*func}在每个离散网格上的积分平均值填入\texttt{target}。
    \end{enumerate}
\end{itemize}
\section{class GhostFiller}
\begin{itemize}
    \item 利用已知的边界条件填充\texttt{Ghost Cell}。
    \item \textbf{模板:} \texttt{template<int Dim>:}
    
    \texttt{Dim}表示空间维数。
    \item \texttt{public:}
    
    \texttt{enum Type\{ Dirichlet=0, Neumann=1 \};}

    \texttt{Type}表示边值条件的种类是\texttt{Dirichlet}或\texttt{Neumann}。
    \item \textbf{成员变量:}
    \begin{enumerate}
        \item \texttt{RectDomain<Dim> domain:} \textbf{private成员变量},表示需要填充\texttt{Ghost Cell}的区域。
        \item \texttt{Tensor<Real,Dim> bdryCond:} \textbf{private成员变量},表示边界处的面平均信息。
        \item \texttt{int nGhost:} \textbf{private成员变量},表示需要填充的\texttt{Ghost Cell}层数。
    \end{enumerate}
    \item \textbf{成员函数:}
    \begin{enumerate}
        \item \texttt{Real FaceAverage(const rVec\& lo,const rVec\& hi,\\const VectorFunction<Dim>* func) const;}
        
        \textbf{private成员函数}

        \textbf{输入:} \texttt{lo}表示控制面的左下角,\texttt{hi}表示控制面的右上角,\texttt{func}指向需要求积分的函数。

        \textbf{输出:} 函数\texttt{func}在控制面\texttt{[lo,hi]}上的积分平均值。

        \textbf{作用:} 计算函数在控制面上的积分平均。

        \item \texttt{GhostFiller(const RectDomain<Dim>\& aDomain, Type atype, int numGhost);}
        
        \textbf{public成员函数}

        \textbf{输入:} \texttt{aDomain}表示需要进行\texttt{Ghost Cell}填充的区域,\texttt{atype}表示边界条件的种类, 
        
        \texttt{numGhost}表示需要填写\texttt{Ghost Cell}的层数。

        \textbf{作用:}构造函数,记录区域信息,边界条件信息,并初始化\texttt{bdryCond}。

        \item \texttt{void fillBdry(const VectorFunction<Dim>* func);}
        
        \textbf{public成员函数}

        \textbf{输入:} \texttt{func}表示边值条件。

        \textbf{作用:}利用边值条件计算区域边界控制面的平均值,并填充至\texttt{bdryCond}。

        \item \texttt{void fillGhost(Tensor<Real,Dim>\& res) const;}
        
        \textbf{输入:} \texttt{res}表示需要填充\texttt{Ghost Cell}张量。

        \textbf{作用:}利用\texttt{res}中原有的体平均值和计算出的面平均值,向外填充\texttt{Ghost Cell},更新\texttt{res}的值。
    \end{enumerate}
\end{itemize}
\end{document}