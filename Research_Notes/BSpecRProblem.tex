\documentclass[UTF8]{ctexart}
\usepackage{ctex}
\usepackage{amsmath}
\usepackage{amsthm}
\usepackage{geometry}
\geometry{left=2.5cm,right=2.5cm,top=2.5cm,bottom=2.5cm}
\usepackage{amssymb}
\usepackage{indentfirst}
\usepackage{graphicx}
\usepackage{subfigure}
\usepackage{listings}
\usepackage{xcolor}
\usepackage{float}
\usepackage{algorithm}  
\usepackage{algorithmicx}  
\usepackage{longtable}
\usepackage{fancyhdr}
\usepackage{appendix}
\usepackage{enumitem}
\usepackage{abstract}
\usepackage{multirow}
\pagestyle{fancy}
\lfoot{}%这条语句可以让页码出现在下方
\theoremstyle{plain}
\newtheorem{thm}{Theorem}[section]
\newtheorem{lem}[thm]{Lemma}
\newtheorem{prop}[thm]{Proposition}
\newtheorem{cor}[thm]{Corollary}

\theoremstyle{definition}
\newtheorem{defn}{Definition}[section]

\theoremstyle{remark}
\newtheorem*{rem}{Remark}
\newtheorem{eg}{Example}[section]
\newcommand{\supp}{\text{supp}}
\newcommand{\Rn}{\mathbb{R}^{n}}
\newcommand{\dif}{\mathrm{d}}
\newcommand{\avg}[1]{\left\langle #1 \right\rangle}
\newcommand{\difFrac}[2]{\frac{\dif #1}{\dif #2}}
\newcommand{\pdfFrac}[2]{\frac{\partial #1}{\partial #2}}
\newcommand{\OFL}{\mathrm{OFL}}
\newcommand{\UFL}{\mathrm{UFL}}
\newcommand{\fl}{\mathrm{fl}}
\newcommand{\op}{\odot}
\newcommand{\cp}{\cdot}
\newcommand{\Eabs}{E_{\mathrm{abs}}}
\newcommand{\Erel}{E_{\mathrm{rel}}}
\newcommand{\DR}{\mathcal{D}_{\widetilde{LN}}^{n}}
\newcommand{\add}[2]{\sum_{#1=1}^{#2}}
\newcommand{\innerprod}[2]{\left<#1,#2\right>}
\newcommand\tbbint{{-\mkern -16mu\int}}
\newcommand\tbint{{\mathchar '26\mkern -14mu\int}}
\newcommand\dbbint{{-\mkern -19mu\int}}
\newcommand\dbint{{\mathchar '26\mkern -18mu\int}}
\newcommand\bint{
{\mathchoice{\dbint}{\tbint}{\tbint}{\tbint}}
}
\newcommand\bbint{
{\mathchoice{\dbbint}{\tbbint}{\tbbint}{\tbbint}}
}
\title{BSpecRProblem Notes}
\author{Shuang Hu}
\begin{document}
\maketitle
\section{文档中的一些小错误}
\begin{itemize}
    \item $(11)$式的上标应当为$n+1$,而不是$n$。
    \item $(25)$式中,矩阵$B=(I+rA)^{-1}(I+rS)$。同理,后面关于$B^{-1}$的讨论也有同样的问题。
    \item $(31)$式下面说明,范数$\|\cdot\|$为$\infty$范数,但Section 1.2.1中实际讨论的是2-范数(谱半径),那么是不是意味着我们没必要强调该范数是无穷范数?
    \item $(35)$和$(36)$式的符号都有小问题,应当是$|\rho(B)\le 1|$, $|\lambda(B^{-1})|=|1+\lambda(Q)|\ge 1$.
\end{itemize}
\section{关于算法的一些疑问}
\begin{itemize}
    \item 注意到,本次采用的虽然是熟悉的有限差分$MOL$算法,但取点的方案却和经典的有限差分算法有较大区别。记$n$为小区间段个数,$h=\frac{1}{n}$是每一段小区间的长度,之前我们学过的有限差分算法求解的是$x_{i}=ih$处函数的近似值,在处理边界条件时为方便起见,往往取$x_{-1}=-h$和$x_{n+1}=(n+1)h$为\texttt{Ghost Cell}。但在本问题的算法中,我们关心的是每段小区间中点处函数的近似取值,即$x_{i}=(i+\frac{1}{2})h$。相应的,\texttt{Ghost Cell}取为$x_{-1}:=-\frac{1}{2}h$和$x_{n}:=1+\frac{1}{2}h$。这样处理之后,方便了\textbf{Neumann边值条件}的处理,但与此同时对\textbf{Dirichlet边值条件}的处理造成了不便。这种处理方式和我们常用的处理方式相比较,优势和劣势分别体现在何处?
    \item 按照设定,$x_{I}$点左侧为较快的物理过程,而$x_{I}$右侧较慢。那么,针对两种不一样的物理过程,为什么采用同样的数值算法进行处理?可能在$x_{I}$左侧采用中心差分和向后欧拉,在$x_{I}$右侧采用中心差分和向前欧拉,会是更合理的处理方式。
    \item 如上一点所述,对于$x_{I}$左侧和右侧的数据点分别求解,其原因是系数$\nu(x)$在$x_{I}$左侧和右侧存在截然不同的表现。如果$\nu$在问题区域$[0,1]$上为一个常值,那么直接在整个区域上进行有限差分离散就可以了。但在稳定性讨论时,我们则是针对一个经典的热方程进行讨论,这是因为我们需要先保证该算法求解热方程的正确性吗?
    \item 注意到,式$(24)$的矩阵$A$并非一个分块对角矩阵,这也就意味着$x_{I}$的左右两侧并没有完全解耦。这或许是稳定性分析的主要难点?
    \item 如果考虑到$x_{I}$左侧和右侧$\nu$的显著区别,那么$r:=\frac{k\nu}{h^2}$至少需要分两段讨论,即$x_{I}$左侧和$x_{I}$右侧的$r$会存在显著区别,这样等式$(19)$则不成立,问题会更为复杂一些。
\end{itemize}
\section{一些思路}
\begin{itemize}
    \item 利用\textbf{Sherman-Morrison formula}计算$B$然后直接分析的思路我认为不能优先考虑。因为表达式$(40)$过于复杂,很难通过$(40)$直接分析矩阵$B$的相关信息。
\end{itemize}
\end{document}