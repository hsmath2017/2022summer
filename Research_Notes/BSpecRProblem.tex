\documentclass[UTF8]{ctexart}
\usepackage{ctex}
\usepackage{amsmath}
\usepackage{amsthm}
\usepackage{geometry}
\geometry{left=2.5cm,right=2.5cm,top=2.5cm,bottom=2.5cm}
\usepackage{amssymb}
\usepackage{indentfirst}
\usepackage{graphicx}
\usepackage{subfigure}
\usepackage{listings}
\usepackage{xcolor}
\usepackage{float}
%\usepackage{algorithm}  
%\usepackage{algorithmicx} 
\usepackage[ruled,linesnumbered]{algorithm2e} 
\usepackage{longtable}
\usepackage{fancyhdr}
\usepackage{appendix}
\usepackage{enumitem}
\usepackage{abstract}
\usepackage{multirow}
\pagestyle{fancy}
\lfoot{}%这条语句可以让页码出现在下方
\theoremstyle{plain}
\newtheorem{thm}{Theorem}
\newtheorem{lem}[thm]{Lemma}
\newtheorem{prop}[thm]{Proposition}
\newtheorem{cor}[thm]{Corollary}

\theoremstyle{definition}
\newtheorem{defn}{Definition}[section]

\theoremstyle{remark}
\newtheorem*{rem}{Remark}
\newtheorem{eg}{Example}[section]
\newcommand{\supp}{\text{supp}}
\newcommand{\Rn}{\mathbb{R}^{n}}
\newcommand{\dif}{\mathrm{d}}
\newcommand{\avg}[1]{\left\langle #1 \right\rangle}
\newcommand{\difFrac}[2]{\frac{\dif #1}{\dif #2}}
\newcommand{\pdfFrac}[2]{\frac{\partial #1}{\partial #2}}
\newcommand{\OFL}{\mathrm{OFL}}
\newcommand{\UFL}{\mathrm{UFL}}
\newcommand{\fl}{\mathrm{fl}}
\newcommand{\op}{\odot}
\newcommand{\cp}{\cdot}
\newcommand{\Eabs}{E_{\mathrm{abs}}}
\newcommand{\Erel}{E_{\mathrm{rel}}}
\newcommand{\DR}{\mathcal{D}_{\widetilde{LN}}^{n}}
\newcommand{\add}[2]{\sum_{#1=1}^{#2}}
\newcommand{\innerprod}[2]{\left<#1,#2\right>}
\newcommand\tbbint{{-\mkern -16mu\int}}
\newcommand\tbint{{\mathchar '26\mkern -14mu\int}}
\newcommand\dbbint{{-\mkern -19mu\int}}
\newcommand\dbint{{\mathchar '26\mkern -18mu\int}}
\newcommand\bint{
{\mathchoice{\dbint}{\tbint}{\tbint}{\tbint}}
}
\newcommand\bbint{
{\mathchoice{\dbbint}{\tbbint}{\tbbint}{\tbbint}}
}
\title{BSpecRProblem Notes}
\author{Shuang Hu}
\begin{document}
\maketitle
\textbf{注:如无特殊说明,本文中的公式编号均为文档BSpecRProblem.pdf中的编号,“文档”也特指BSpecRProblem.pdf。}
\section{文档中的一些小问题}
\begin{itemize}
    \item 图1上面的叙述中,$x_{i}=ih$存在误导,文档所述算法的空间离散是在区间中点处进行而非区间端点。
    \item $(11)$式的上标应当为$n+1$,而不是$n$。
    \item $(25)$式中,矩阵$B=(I+rA)^{-1}(I+rS)$。后面关于$B^{-1}$的讨论也有同样的问题。
    \item $(35)$和$(36)$式的符号都有小问题,应当是$|\rho(B)|\le 1$, $|\lambda(B^{-1})|=|1+\lambda(Q)|\ge 1$。
\end{itemize}
\section{关于算法的一些疑问}
\begin{itemize}
    \item 注意到,本次采用的虽然也是有限差分$MOL$算法,但取点的方案却和经典的有限差分算法有较大区别。记$n$为小区间段个数,$h=\frac{1}{n}$是每一段小区间的长度,之前我们学过的有限差分算法求解的是$x_{i}=ih$处函数的近似值,在处理边界条件时为方便起见,往往取$x_{-1}=-h$和$x_{n+1}=(n+1)h$为\texttt{Ghost Cell}。但在本问题的算法中,我们关心的是每段小区间中点处函数的近似取值,即$x_{i}=(i+\frac{1}{2})h$。相应的,\texttt{Ghost Cell}取为$x_{-1}:=-\frac{1}{2}h$和$x_{n}:=1+\frac{1}{2}h$(文中列方程时实际上消去了\texttt{Ghost Cell})。这种处理方案,相比于传统的有限差分方法,有哪些优势?在本问题中为何要采用这种处理方案?
    \item 按照设定,$x_{I}$点左侧为较快的物理过程,而$x_{I}$右侧较慢。那么,针对两种不一样的物理过程,为什么采用同样的数值算法进行处理?可能在$x_{I}$左侧采用中心差分和向后欧拉,在$x_{I}$右侧采用中心差分和向前欧拉,会是更合理的处理方式。
    % \item 如上一点所述,对于$x_{I}$左侧和右侧的数据点分别求解,其原因是系数$\nu(x)$在$x_{I}$左侧和右侧存在截然不同的表现。如果$\nu$在问题区域$[0,1]$上为一个常值,那么直接在整个区域上进行有限差分离散就可以了。但在稳定性讨论时,我们则是针对一个经典的热方程进行讨论,这是因为我们需要先保证该算法求解热方程的正确性吗?
    \item 注意到,式$(24)$的矩阵$A$并非一个分块对角矩阵,这也就意味着$x_{I}$的左右两侧并没有完全解耦。这或许是稳定性分析的主要难点?
    \item 如果考虑到$x_{I}$左侧和右侧$\nu$的显著区别,那么$r:=\frac{k\nu}{h^2}$至少需要分两段讨论,即$x_{I}$左侧和$x_{I}$右侧的$r$会存在显著区别,问题会更为复杂一些。
\end{itemize}
\section{稳定性分析问题简述}
假设$A$和$S$分别按$(24)$, $(23)$式定义,记$B:=(I+rA)^{-1}(I+rS)$,文档中所述算法数值稳定等价于式$(31)$成立,即序列$b_{n}:=\|B^{n}\|_{\infty}$为一个有界序列。而该问题可以转化为证明矩阵$B$的谱半径$\rho(B)<1$。

由于$B$的表达式较难给出,因此我们把问题转化为研究$B^{-1}$的模最小特征值。如果能证明$B^{-1}$的模最小特征值的模长大于1,那么也就说明了$B$的谱半径小于1。按文中$(16)$式,取$a_{11}=3,a_{12}=-1$,可得$B^{-1}=I+Q$,其中
\begin{equation}
    Q=r\begin{bmatrix}
        3&-1& & & & & & & & \\
        -1&2&-1& & & & & & &\\
        &\ddots&\ddots&\ddots&&&&&&\\
         & & &-1&1+\frac{1}{1+r^2}&\frac{2r^2-1}{1+r^2}&\frac{-3r^2}{1+r^2}&\frac{r^2}{1+r^2}&&\\
         & & & &-\frac{1}{1+r^2}&1+\frac{1-2r^2}{1+r^2}&\frac{2r^2-1}{1+r^2}&-\frac{r^2}{1+r^2}& \\
         & & & & &-1&2&-1& & \\
         & & & & &  &\ddots&\ddots&\ddots& \\
         & & & & & & &-1&2&-1\\
         & & & & & & &  &-1&1\\
    \end{bmatrix}.
\end{equation}
推导过程见文中等式$(34)$。按这种方式,问题可以转化为证明矩阵$Q$的所有特征值实部大于0。但这是一个充分而非必要的条件,事实上,只要$Q$的所有特征值$\lambda(Q)$满足:
\begin{equation}
    |1+\lambda(Q)|\ge 1,
\end{equation}
即可说明稳定性结论成立。
\section{参考思路的可行性分析}
文档第二部分给了一些参考思路,在本部分中,我对这些思路的可行性进行分析和评述。
\subsection{秩1校正思路}
该部分的主要思想是把待分析的矩阵$Q$分解为$Q=Q_{1}+Q_{2}$,其中$Q_1$是三对角正定矩阵,$Q_2$是秩1矩阵,它们的表达式分别为:
\begin{equation}
    \label{eq:rank1repair}
    Q_{1}=r\begin{bmatrix}
        3&-1& & &\\
        -1&2&-1& & \\
         &\ddots&\ddots&\ddots& \\
         & &-1&2&-1\\
         & & &-1&1\\
    \end{bmatrix}
    ,Q_{2}=\begin{bmatrix}
        \mathbf{O}& & \\
        & Q_{4\times 4}& \\
        & & \mathbf{O}\\
    \end{bmatrix}
    ,Q_{4\times 4}=\frac{r^3}{1+r^2}\begin{bmatrix}
        -1&3&-3&1\\
        1&-3&3&-1\\
        0&0&0&0\\
        0&0&0&0\\
    \end{bmatrix}.
\end{equation}
(同文档中式(37))。由于我们只关心矩阵$Q$的特征值实部是否大于0,且参数$r>0$,我们可以把公因子$r$提取出来,转化为分析矩阵$P:=P_{1}+\frac{r^{2}}{1+r^{2}}P_{2}$的特征值。$P_{1}$,$P_{2}$的表达式分别为:
\begin{equation}
    \label{eq:rank1repairP}
    P_{1}=\begin{bmatrix}
        3&-1& & &\\
        -1&2&-1& & \\
         &\ddots&\ddots&\ddots& \\
         & &-1&2&-1\\
         & & &-1&1\\
    \end{bmatrix}
    ,P_{2}=\begin{bmatrix}
        \mathbf{O}& & \\
        & P_{4\times 4}& \\
        & & \mathbf{O}\\
    \end{bmatrix}
    ,P_{4\times 4}=\begin{bmatrix}
        -1&3&-3&1\\
        1&-3&3&-1\\
        0&0&0&0\\
        0&0&0&0\\
    \end{bmatrix}.    
\end{equation}
$P_{1}$的形式比较接近离散拉普拉斯算子的形式,分析其特征值相对比较容易,而$P_{2}$是一个秩1矩阵,也比较易于分析。对于$P=P_{1}+\frac{r^{2}}{1+r^2}P_{2}$,主要的难点在于$P_{2}$前面的系数与参数$r$相关。但在分析$P$的特征值时,我们依旧可以利用$P_{1}$和$P_{2}$的一些比较好的性质。该思路是我目前的首选思路,下一节的叙述中也会围绕这一条展开。
\subsection{利用Sherman-Morrison 公式给出矩阵$B$的表达式}
我认为这一条思路可行性较低。如$(40)$式所示,$B$的表达形式过于复杂,且涉及到求两个较复杂矩阵减法的过程,以至于我们很难对矩阵$B$的特征多项式进行分析。
\subsection{矩阵的半正定分解}
如果我们可以把矩阵$Q$分解为两个半正定矩阵的和,则问题即刻得到解决。但这条思路目前也不是我的首选,原因如下:
\begin{itemize}
    \item $Q$非对称矩阵,而一般讨论正定矩阵都需要在对称阵的框架下进行。在这种情况下,我们需要分解的矩阵不是$Q$,而是更为复杂的$\frac{Q+Q^{t}}{2}$。
    \item 数学问题中“举例说明存在”很多时候并非一件容易的事情,特别是在该问题中,$Q$的大小,矩阵块$Q_{4\times 4}$的位置和$r$的取值都是需要考虑的变量。
\end{itemize}
\subsection{考虑其他(易于计算的)矩阵范数}
文档中表1已经表明了直接计算矩阵$B$的$\infty$范数不可行,而矩阵范数除了$\infty$范数和$1$范数外都并不易于计算。所以这条思路也不太可行。
\section{我的思路}
由于$r\in(0,\infty)$,记$t=\frac{r^2}{1+r^2}$,可知$t\in(0,1)$,我们需要分析的矩阵$P(t)=P_{1}+\frac{r^{2}}{1+r^2}P_{2}=P_{1}+tP_{2}$。

首先我们讨论实特征值。分析$P(t)$的特征多项式$f(\lambda,t):=\det(\lambda I-P(t))$。为方便讨论,考虑多项式$g(\lambda,t):=\det(P(t)-\lambda I)$,该多项式的根和$f(\lambda,t)$相同。由于$\text{rank}(Q_{2})=1$,由行列式的性质可知$g(\lambda,t)$关于$t$是一次多项式,从而$\pdfFrac{g}{t}= C(\lambda)$,$C(\lambda)$仅与$\lambda$相关。这意味着对于给定的$\lambda\in\mathbb{R}$,$g(\lambda,t)$关于$t$单调。

\begin{thm}
    \label{thm1}
    给定$\lambda_{0}\in\{\omega\in\mathbb{R}:C(\omega)\neq 0\}$,存在$t_{0}\in (0,\tau)$ $(\tau>0)$使得$\lambda_{0}$为$P(t_{0})$的特征值,当且仅当$g(\lambda_{0},0)g(\lambda_{0},\tau)<0$。
\end{thm}
\begin{proof}
    $"\Leftarrow":$由于$h(t):=g(\lambda_{0},t)$为连续函数,且$h(0)h(\tau)<0$,由零点存在定理,$\exists t_{0}\in(0,\tau)$ s.t. $h(t_{0})=0$,即$g(\lambda_{0},t_{0})=0$,从而$\lambda_{0}$是$P(t_{0})$的特征值。

    $"\Rightarrow":$由于$\pdfFrac{g(\lambda_{0},t)}{t}=C(\lambda_{0})$,$h(t)$是一次函数。且由于$C(\lambda_{0})\neq 0$,可知$h$非常值函数。故$h(t)=C(\lambda_{0})(t-t_{0})$,计算可得:
    \begin{equation}
        g(\lambda_{0},0)g(\lambda_{0},\tau)=C(\lambda_{0})^{2}(\tau-t_{0})(-t_{0})<0.
    \end{equation}
    最后一步不等式的依据是$0<\tau<t_{0}$。
\end{proof}
\begin{thm}
    \label{thm2}
    给定$\lambda_{0}\in\{\omega\in\mathbb{R}:C(\omega)=0\}$,存在$t_{0}\in(0,\tau)$ $(\tau>0)$使得$\lambda_{0}$为$P(t_{0})$的特征值,当且仅当$g(\lambda_{0},0)=g(\lambda_{0},\tau)=0$。
\end{thm}
\begin{proof}
    记$h(t):=g(\lambda_{0},t)$,当$C(\lambda_{0})=0$时,
    \begin{equation}
        \difFrac{h(t)}{t}=\pdfFrac{g(\lambda_{0},t)}{t}=C(\lambda_{0})=0,
    \end{equation}
    从而$h(t)\equiv h(0)$。因此,在这种情况下,$h(t_{0})=0\Leftrightarrow h(0)=h(\tau)=0$,即$g(\lambda_{0},0)=g(\lambda_{0},\tau)=0$。
\end{proof}
\begin{rem}
    \begin{equation}
        g(\lambda,0)=g(\lambda,1)\Leftrightarrow C(\lambda)=0.
    \end{equation}
\end{rem}
\begin{cor}
    给定$\tau>0$,如果$P(\tau)$的所有实特征值都大于0,那么$\forall t\in[0,\tau]$,$P(t)$的所有实特征值都大于0。
\end{cor}
\begin{proof}
    由于$P(0)=P_{1}$为对称正定矩阵,$\forall\lambda\le 0$,$g(\lambda,0)>0$。根据$g(\lambda,t)$的定义,$\lim_{\lambda\rightarrow -\infty}g(\lambda,\tau)=+\infty$,且$g(\lambda,\tau)$关于$\lambda$连续。又根据已知条件,$\forall\lambda\le0$,$g(\lambda,\tau)\neq 0$,由零点存在性定理可知:
    \begin{equation}
        \forall\lambda\le 0,g(\lambda,\tau)>0.
    \end{equation}
    结合定理\ref{thm1}和定理\ref{thm2},$\forall\lambda\le 0$, $t\in[0,\tau]$, $g(\lambda,t)>0$。即$P(t)$的所有实特征值均大于0。
\end{proof}

由推论3可知,如果我们可以验证$P(1)$的所有实特征值均大于0,则说明$\forall t\in[0,1],P(t)$的所有实特征值均大于0。

如果$P(1)$存在非正特征根,则利用二分查找算法寻找$t_{0}$,使得$\forall t\in(0,t_{0})$,$P(t)$不存在非正特征根。算法流程大致如下:

\begin{enumerate}
    \item 设定$t_{0}=0$,$t_{e}=1$,最小查询区间长度$\epsilon$。
    \item 如果$|t_{e}-t_{0}|<\epsilon$,终止程序并返回$t_{0}$。
    \item 计算$mid=\frac{t_{0}+t_{e}}{2}$。
    \item 判断多项式$g(\lambda,mid)$是否存在非正实数根。如果存在,设定$t_{e}=mid$;如果不存在,设定$t_{0}=mid$,返回第二步。
\end{enumerate}

推论3保证了二分法输出的$t_{0}$满足:$\forall t\in[0,t_{0}]$,$P(t)$不存在非正的特征根。

这样我们可以保证在$\frac{r^2}{1+r^2}\in[0,t_{0}]$的条件下矩阵$B$不存在绝对值大于等于1的实特征根。

\section{存在问题和研究展望}

目前这个思路可能存在下面的这些难点:
\begin{itemize}
    \item 给定$\tau\in(0,1]$,不容易计算出$P(\tau)$特征值的解析表达式,如果找不到合适的估计,则需要考虑数值求解矩阵特征值。
    \item $P(\tau)$的最小实特征值与矩阵$P$的大小$N$,以及矩阵块$P_{4\times 4}$的装配位置$M$有关。
    \item 由于复数域并非有序域,且第五部分所述方案高度依赖零点存在定理,这个方案不能直接用于解决复特征根的问题。$P(0)$是实对称矩阵,特征根均为实数,但对于$\tau\neq 0$,$P(\tau)$可能出现实部非正的复特征根,这是目前的方案无法探测的。
\end{itemize}

研究展望:
\begin{itemize}
    \item 试着给出$P(1)$特征值的一个下界估计。
    \item 检查$P(1)$是否存在非实数特征根。如果存在,则设计一组方案处理之。
    \item 完成该算法求解热方程的稳定性分析后,尝试对区域耦合的扩散方程进行稳定性分析。
\end{itemize}
\end{document}